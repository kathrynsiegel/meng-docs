\chapter{Previous Work} Since Leo Breiman introduced random forests to the data
science community in 2001 \cite{Breiman}, researchers have tested various
refinements of the algorithm. One area of focus has been augmenting the
algorithm for online learning. This chapter describes the existing literature
on incremental and online random forests.

\section{Ensemble models} A random forest is a type of ensemble model, which
average the predictions of many different ``reasonably good'' models to produce
a prediction that better estimates the true hypothesis. Ensemble models are
highly successful as machine learning tools, because they avoid the
chance-dependent pitfalls of many singular models. For example, gradient
descent methods can get stuck in local minima, but combining many models
increases the chance that one will find the global minimum. Alternatively, even
if none of the models in the ensemble  produce the true hypothesis, averaging
every prediction can lead to a a prediction that more closely matches the
underlying truth. \cite{Dietterich}

There are several established methods for aggregating model predictions in
ensembles. The Bayesian voting algorithm iterates through all hypotheses
produced by models in the ensemble, and then combines the results based on how
likely the hypothesis is given the sample. Another method is to manipulate the
training data via bagging, known as bootstrap aggregation, and boosting, known
as weighted training. Bagging involves randomly sampling points with
replacement from a common dataset; the default Spark implementation of random
forests uses bagging over boosting. Other ensemble methods exist but are not
utilized in Spark ML.


\section{Online random forests} Standard random forests are offline
classifiers; trees are built on static datasets. Offline classifiers are poorly
suited for datasets for which additional data points become available
incrementally. A prevalent example is logging; logs are published as actions
take place within a system, and a classifier for these logs must take into
account the new information to avoid becoming inaccurate. Similar problems
requiring an online classifier are found throughout industry. For example,
Yahoo uses an online classifier to characterize relevant articles to show each
user on its homepage. In many of these use cases, retraining the classifier
from scratch would take an excessive amount of time, given the enormous amount
of existing training data. Since new batches of data are often far smaller than
the size of the aggregate training data, the new data only shifts a
classifier's behavior slightly. Retraining an entire classifier from scratch to
capture slight shifts seems wasteful; online classifiers provide a far more
efficient approach.

In this thesis, I study incremental random forest classifiers--forests that
update themselves with new batches of data. Incremental forests are a subset
online random forests; they are updated with new batches of data, rather than
with one new point at a time. Batched updates can take advantage of Spark's
strength, which is distributed, batched computation. Simply implementing an
online random forest classifier in Spark would be computationally wasteful, as
it would require a new Spark job for every additional point. Focusing on
batched updates plays to Spark's strengths and addresses a sparser area of the
machine learning literature. Existing online methods are easily extended into
batched methods, and batched computation provides an opportunity for
optimizations.

\subsection{Saffari online random forests} The most well-known implementation
of an online random forest is the Saffari implementation. \cite{Saffari} In
offline mode, a Saffari random forest behaves like a typical random forest:
each tree is created by sampling the original training set, the split at each
node is chosen as the best among a random set of feature candidates, and
predictions are made by summing conditional probabilities among all of the
trees.

In online mode, trees receive a serial string of points and are grown in an
extremely randomized fashion; at every node, tests and thresholds are chosen
randomly. Specifically, when a node is created, it establishes a set of N
random tests and maintains statistics on the left and right partitions created
by each test. When a new point is added to the tree, the point's features place
it in a leaf node. The algorithm then recalculates the gain $\mathrm{G_n}$ with
respect to each test in that node $n$ according to the following equation:
\begin{equation} \mathrm{G_n}=\ell_n - \frac{|samples_l|}{|samples_n|}*\ell_l -
\frac{|samples_r|}{|samples_n|}*\ell_r, \end{equation} where $\ell$ indicates
loss, $l$ represents the left partition of a split, and $r$ represents the
right partition of a split. The Saffari algorithm splits a leaf based on two
hyperparameters: $\alpha$, the minimum number of samples a node must see, and
$\beta$, the minimum gain a split must achieve. When both conditions are
satisfied, the node splits.

The Saffari algorithm also specifies that trees can be discarded randomly,
where the probability that a tree is discarded increases with its out-of-bag
error. This trait allows the random forest to adapt to changes in the data
distribution.

Saffari claims this algorithm is better than alternatives such as the Hoeffding
tree algorithm (addressed later in this chapter), because it fits better to the
inherent nature of decision trees. Empirically, Saffari random forests perform
better than boosted forests. In my research, I test the efficacy of using the
extremely randomized trees of the Saffari algorithm. I also experiment with the
aforementioned split hyperparameters, tweaking my algorithm to select more
optimal gain and points thresholds. Since the Spark random forest
implementation closely matches the Saffari offline random forest
implementation, many of my initial modifications to the Spark code to provide
online learning capability are in line with the Saffari online-mode algorithm.

\subsection{Denil online random forests} In "Consistency of Online Random
Forests," Misha Denil et al. proposes and evaluates improvements to the Saffari
online random forest algorithm. Denil online forests partition the sequence of
data points into ``structure'' points and ``estimation'' points. \cite{Denil}
Structure points influence the structure of the tree but do not affect the
predictions made in tree leaves. Estimation points do not influence the
structure of the tree, but are used to re-estimate probabilities. The Denil
algorithm uses the same split selection procedure as the Saffari algorithm. The
paper shows that this refined implementation achieves a higher accuracy on
complex datasets compared to a comparable implementation of the Saffari
algorithm.

Much of the existing literature around online random forests acknowledges one
primary setback: due to decision trees’ recursive structure, lower data cannot
be used to correct earlier decisions. Both the Saffari and the Denil
implementations have this flaw. The new few subsections address algorithms that
involve regrowing part or all of select decision trees within a random forest.
My research draws on both classes of techniques-{}-incremental growth and
regeneration.

\subsection{Mondrian forests}

Another type of online random forest is a Mondrian forest \cite{NIPS2014_5234},
which are comprised of augmented extremely randomized decision trees. The
algorithm for building each Mondrian decision tree is as follows.  We start at
the root and allocate a budget, $\lambda$, for this node. Then, we recursively
process each node by randomly choosing split locations on the ranges of feature
values. For each point $j$, dimension $d$, and dimension-wise maximum and
minimum $u_{jd}$ and $l_{jd}$, let $E = \sum_d(u_{jd}-l_{jd})$. Then, these
cuts each cost $\lambda' = \lambda - E$. If we can ``afford'' the cut, we split
the node and assign budget $\lambda'$ to each subinterval. The tree stops
growing when no more cuts can be afforded at any leaf node. Throughout the
growing process, we denote a split hierarchy using a time variable. A node's
time variable $\tau$ is set when a split is made within that node; the value of
this variable is equivalent to $E + \tau_{parent}$. 

To adapt Mondrian trees for online learning, we use the time parameter to
determine where a new cut should be inserted. Starting at the root, we recurse
down the tree until the cost coefficient $E_{new}$ for this new node is less
than the cost coefficient $E_{old}$ for some node. We then insert the new node
as the parent of this old node and adjust all children nodes accordingly.

Mondrian forests achieve an accuracy very close to offline Breiman random
forests and extremely randomized forests trained on the same fraction of the
data. The Mondrian algorithm significantly outperforms the Saffari algorithm
when trained in online mode on the same fraction of data. Furthermore, Mondrian
forests were shown to adapt to new data an order of magnitude faster than
simply regrowing a forest from scratch.

The drawbacks of Mondrian forests include its relative intolerance of
irrelevant data. Because splits are random, irrelevant and relevant features
are equally likely to be chosen; noisy features can then harm the overall
accuracy of each tree. Additionally, a Mondrian forest in online mode might
choose to insert a node close to a root, thus initiating recalculation of a
large section of the tree.

While Mondrian forests outperform Saffari online random forests, regrowing a
tree section requires passing through all of the data points yet seen. As
datasets grow larger, the overhead incurred by additional passes through the
data will outweight the efficiency gains of the algorithm. Mondrian forests
have only been shown as more efficient than Saffari online random forests on
datasets of a few thousand data points. As such, the optimal strategy should
regrow parts of a random forest but should not require multiple passes over the
data.

\section{Concept drift} Concept drift describes changes in data distribution in
an online learning setting that cause the mathematical relationships between
the input variables and output predictions to change. These shifts cause ML
classifiers trained on earlier data points to become inaccurate. The online
random forest algorithms discussed earlier in this chapter adapt poorly to
concept drift, as they accommodate new data by splitting leaves or regrowing
small sections of trees in the forest. Concept drift can cause splits in nodes
higher in the trees to become inaccurate, and a poor decision higher in the
tree more significantly impacts performance than a mistake closer to the
bottom. Online random forests adapt more poorly than other online classifiers
to concept drift, as splits made in nodes are essentially permanent. In
contrast, classifiers that use, for example, linear or logistic regression
could just shift internal weights until the concept drift is accounted for.

Therefore, accounting for concept drift in online random forests requires
regrowing trees--the decisions from these trees must counteract bad decisions
from other trees resulting from wrong splits. Purely incremental strategies, or
those that just split leaves, should not be the only methods by which online
random forests adapt to change.

\section{Combined strategies} Several papers in the literature use Hoeffding
trees to grow online random forests. Hoeffding trees for online learning were
first proposed in a paper by Domingos and Hulten; when growing, these trees
maintain several candidate splits in each leaf, with the quality of each split
estimated in an online manner. \cite{Domingos} In contrast with the minimum
gain parameter controlling splits in Denil and Saffari random forests,
Hoeffding trees use a measure of the Hoeffding bound to ensure that a split is
optimal. Hoeffding trees split leaves when the Hoeffding bound indicates that
the current best split is the optimal split, within reasonable certainty. 

A paper by Bifet et al. describes an implementation of the Hoeffding tree
algorithm that adapts to concept drift. \cite{Bifet} The algorithm grows
Hoeffding trees of different sizes; the authors reason that smaller trees can
adapt more quickly to concept drift, whereas larger trees are less sensitive to
noisy deviations. The trees are occasionally either partly or fully regrown,
depending on a metric called ADWIN that estimates drift. The Bifet paper's
results show that the method is effective on small generated datasets with
concept drift.

Abdusalam et al. developed an algorithm that grows random forests incrementally
by using Hoeffding trees and selecting entire trees for replacement. Unlike the
algorithm from the paper by Bifet et al., the system chunks a data stream, then
processes data chunks serially. The algorithm detects concept drift using a
two-window technique; if a tree's classification error between the two windows
differs by an amount less than a threshold, it is not grown further. Otherwise,
grow the tree incrementally. With every given batch, 25\% of the trees are
automatically regrown, with additional trees regrown if the system detects
concept drift.

I take into account these algorithms when developing my system in Spark.
Namely, my implementation heavily involves a balance between incremental growth
and tree replacement, much like the algorithms in the Bifet and Abdusalam
papers. I refer back to these previously-developed algorithms in my discussion
of the Spark incremental random forests system developed in this thesis.
