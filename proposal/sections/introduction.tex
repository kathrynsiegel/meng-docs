\documentclass[../proposal.tex]{subfiles}

\begin{document}

\section{Introduction}

A data scientist's pipeline for creating and training machine learning (ML)
models is a multi-step, repetitive process. Specifically, the process involves 
defining the model specifications, training the model on a dataset, cross-
validating the results on test data, tweaking the model specifications and/or
dataset, and beginning the whole cycle over again. Eventually, a model is 
produced that meets a reasonably high standard of accuracy in the cross-
validation step, and this model is the one that is selected for further use.
Because of this cyclical iteration pattern, existing machine learning tools
require users to test model hypotheses sequentially. Users must train one model
after the other, manually specifying parameters such as the datafolds and the
number of training iterations. This manual component incurs a significant time
cost. 

WahooML is a system that accelerates this iterative process by optimizing model
training runs and allowing multiple models to be trained together. The
system also exposes a developer API that allows users to specify custom models
and estimators. WahooML revolves around the concept of a ``hypothesis,'' which
refers to an insight that a data scientist is trying to reach by training
models on a set of data. WahooML accepts a full data set and the specifications
of the training model. The query optimizer component of the system then
optimally groups and tests datafolds such that it finds the ideal output in the
least computationally expensive way possible. The ModelDB component of the
system stores intermediate computation, as well as the outputted model, such
that WahooML can strategically reuse past computation if it were to see an
identical or similar model specification in the future. Finally, the WahooML
user interface (UI) displays interesting visualizations of the entire process.

My work thus far has focused on creating a basic system that stores trained
models in an external database, such that results do not have to be recomputed.
My thesis work will mainly focus on the query optimizer component of the
system. In this proposal, I describe the research and decisions that led to the
current design of the system, as well as the optimizations I plan to study and
implement in WahooML.

\end{document}
