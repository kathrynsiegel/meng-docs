\documentclass[../proposal.tex]{subfiles}

\begin{document}

\section{Previous Work}

In this section, I provide an overview of the existing ML tools Weka and
AzureML and evaluate the benefits and shortcomings of each tool. On the whole,
WahooML will provide significant performance advantages over these popular
existing tools.

\subsection{Weka}
Weka is a data mining program written in Java. It provides a GUI through which
a user can import measurements and view them in a table.  Through this GUI,
data scientists can select machine learning classification algorithms and the
parameters associated with these algorithms. One can also select which subset
of features to use with an ML algorithm; there is a supervised learning feature
to help users select attributes. Weka then runs the ML algorithm and reports
detailed results of the model back to the user. Morever, users may write a new
classifier or filter using Weka's Java API and use it alongside the
already-existing tools. [5]

The system has an flexible, intuitive interface that provides ample data for
data scientiests. However, there are several significant drawbacks to using
Weka. Iterating on different feature sets and algorithms can be slow and
cumbersome due to the manual selection process for data. A lot of information
is provided by Weka, but Weka provides no guidance as to how that information
could help training future models. Moreover, the system forces the user to
conduct model refinement sequentially, without any method to avoid unnecessary
recomputation. Overall, there are many components of Weka, such as the
intuitive UI and wealth of outputted data, that we should emulate in WahooML.
However, WahooML would provide benefits such as parallel model training and
strategic storage to reduce overall computation time.

\subsection{Azure ML}
Azure Machine Learning is a system that allows users to run machine learning
workflows. It provides a UI for users as well as cloud hosting and storage.
Users interact with Azure ML by creating dataflow diagrams that detail
experiments. Individual nodes in these experiments represent operations on data
ranging from transformations to model training. Azure ML then processes the
experiment and returns the results and computed statistics to the user. Users
can programmatically create their own modules and import tham into Azure ML. [2]

Similarly to Azure ML, WahooML will provide data pipeline visualizations and
allow users to integrate their custom code with the system. However, Azure ML
limits user capabilities in several ways. The system forces users to use the
Azure cloud instead of a custom hosting site, restricts custom additions to
live within individual modules, and is very slow when compared to alternatives.
WahooML will avoid these shortcomings, while also offering benefits such as
warm-start runs and incremental training capabilities.

\end{document}
